\documentclass[a4paper]{article}
\usepackage[left=1.25in,right=1.25in,top=1in,bottom=1in]{geometry} % word form
\usepackage[usenames,svgnames,hyperref]{xcolor}
\usepackage{amsmath,array,amsbsy,bm,dashbox,fancybox,graphicx,paralist,slashed,longtable,tabularx,booktabs,floatrow,subfig,tensor}
\usepackage{subfiles}
\usepackage{hyperref}
\hypersetup{colorlinks,linkcolor=blue,citecolor=red}
\linespread{1.25}
%\renewcommand{\rmdefault}{ptm}
\usepackage[T1]{fontenc}
\usepackage{newtxtext, newtxmath}


\begin{document}
The coefficients in potential at LO is collected to a matrix, i.e.,
\begin{equation}
	\bordermatrix{
    & K^-p & \Sigma^+\pi^- & \Sigma^0\pi^0 & \bar{K}^0n & \Sigma^-\pi^+ & \Lambda\pi^0 \cr
K^- p & 2 & 1 & \frac{1}{2} & 1 & 0 & \frac{\sqrt{3}}{2} \cr
\Sigma^+\pi^- & 1 & 2 & 2 & 0 & 0 & 0 \cr
\Sigma^0 \pi^0 & \frac{1}{2} & 2 & 0 & \frac{1}{2} & 2 & 0 \cr
\bar{K}^0n & 1 & 0 & \frac{1}{2} & 2 & 1 & -\frac{\sqrt{3}}{2} \cr
\Sigma^-\pi^+ & 0 & 0 & 2 & 1 & 2 & 0 \cr
\Lambda\pi^0 & \frac{\sqrt{3}}{2} & 0 & 0 & -\frac{\sqrt{3}}{2} & 0 & 0 \cr
}.
\end{equation}
The $D_{ij}$ and $L_{ij}$ in the NLO potential are given by
\begin{equation}
	\bordermatrix{
    & K^-p & \Sigma^+\pi^- & \Sigma^0\pi^0 & \bar{K}^0n & \Sigma^-\pi^+ & \Lambda\pi^0 \cr
K^-p & 4(b_0+b_D)m_K^2 & (b_D - b_F)\mu_1^2 & \frac{(b_D-b_F)\mu_1^2}{2} &2(b_D+b_F)m_K^2 & 0 & -\frac{(b_D+3b_F)\mu_1^2}{2\sqrt{3}} \cr
\Sigma^+\pi^- & (b_D-b_F)\mu_1^2 & 4(b_0+b_D)m_\pi^2 & 0 & 0 & 0 & 0 \cr
\Sigma^0\pi^0 & \frac{(b_D-b_F)\mu_1^2}{2} & 0 & 4(b_0+b_D)m_\pi^2 & \frac{(b_D-b_F)\mu_1^2}{2} & 0 & 0 \cr
\bar{K}^0n & 2(b_D+b_F)m_K^2 & 0 & \frac{(b_D-b_F)\mu_1^2}{2} & 4(b_0+b_D)m_K^2 & (b_D-b_F)\mu_1^2 & \frac{(b_D+3b_F)\mu_1^2}{2\sqrt{3}} \cr
\Sigma^-\pi^+ & 0 & 0 & 0 & (b_D-b_F)\mu_1^2 & 4(b_0+b_D)m_\pi^2 & 0 \cr
\Lambda\pi^0 & -\frac{(b_D+3b_F)\mu_1^2}{2\sqrt{3}} & 0 & 0 & \frac{(b_D+3b_F)\mu_1^2}{2\sqrt{3}} & 0 & \frac{4(3b_0+b_D)m_\pi^2}{3} \cr
},
\end{equation}
and,
\begin{equation}
	\bordermatrix{
    & K^- p & \Sigma^+\pi^- & \Sigma^0\pi^0 & \bar{K}^0n & \Sigma^-\pi^+ & \Lambda\pi^0 \cr
K^-p & 2d_2+d_3+2d_4 & -d_1+d_2+d_3 & \frac{-d_1-d_2+2d_3}{2} & d_1+d_2+d_3 & -2d_2+d_3 & \frac{-\sqrt{3}(d_1+d_2)}{2} \cr
\Sigma^+\pi^- & -d_1+d_2+d_3 & 2d_2+d_3+2d_4 & -2d_2+d_3 & -2d_2+d_3 & -4d_2+2d_3 & 0 \cr
\Sigma^0\pi^0 & \frac{-d_1-d_2+2d_3}{2} & -2d_2+d_3 & 2(d_3+d_4) & \frac{-d_1-d_2+2d_3}{2} & -2d_2+d_3 & 0 \cr
\bar{K}^0n & d_1+d_2+d_3 & -2d_2+d_3 & \frac{-d_1-d_2+2d_3}{2} & 2d_2+d_3+2d_4 & -d_1+d_2+d_3 & \frac{-\sqrt{3}(d_1+d_2)}{2} \cr
\Sigma^-\pi^+ & -2d_2+d_3 & -4d_2+2d_3 & -2d_2+d_3 & -d_1+d_2+d_3 & 2d_2+d_3+2d_4 & 0 \cr
\Lambda\pi^0 & \frac{-\sqrt{3}(d_1+d_2)}{2} & 0 & 0 & \frac{-\sqrt{3}(d_1+d_2)}{2} & 0 & 2d_4 \cr
}
\end{equation}








































\end{document}
  


































